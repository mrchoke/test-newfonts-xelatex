\documentclass[titlepage,a4paper,10pt]{scrartcl}
\usepackage{fontspec}[no-math] 
\usepackage{xltxtra}
\usepackage{polyglossia}
\usepackage{ifthen}

\defaultfontfeatures[Garuda,TlwgTypist]{
ItalicFont      = *-Oblique ,
BoldFont        = *-Bold ,
BoldItalicFont  = *-BoldOblique ,
Extension       = .otf ,
Mapping       = tex-text,
Script          = Thai
}

\defaultfontfeatures[Norasi]{
ItalicFont      = *-Italic ,
SlantedFont     = *-Oblique ,
BoldFont        = *-Bold ,
BoldItalicFont  = *-BoldItalic ,
BoldSlantedFont = *-BoldOblique ,
Extension       = .otf ,
Mapping       = tex-text,
Script          = Thai
}

% Laksaman or Saraban
\defaultfontfeatures[Laksaman]{
ItalicFont      = *-Italic ,
BoldFont        = *-Bold ,
BoldItalicFont  = *-BoldItalic ,
Extension       = .otf ,
Mapping       = tex-text,
Script          = Thai
}

% Chulabhorn Likit Text
\defaultfontfeatures[ChulabhornLikitText]{
Path                = ./,
UprightFont   = *-Regular,
BoldFont        = *-Bold ,
Extension       = .otf ,
Mapping       = tex-text,
Script          = Thai
}

% Anakotmai
\defaultfontfeatures[Anakotmai]{
Path                = ./,
UprightFont   = *-Light,
BoldFont             = *-Bold ,
Extension       = .otf ,
Mapping       = tex-text,
Script          = Thai
}

% Thummasat
\defaultfontfeatures[TUFont]{
Path                = ./,
BoldFont             = *-Bold ,
Extension       = .ttf,
Mapping       = tex-text,
Script          = Thai
}

% Pridi
\defaultfontfeatures[Pridi]{
Path                = ./,
UprightFont   = *-Regular,
BoldFont             = *-Bold ,
Extension       = .ttf,
Mapping       = tex-text,
Script          = Thai
}

\newfontfamily\laksaman{Laksaman}
\newfontfamily\ChulabhornLikitText{ChulabhornLikitText}[
  FontFace = {mb}{n}{ Font = *-Medium },
  FontFace = {l}{n}{ Font = *-Light },
]

\newfontfamily\Anakotmai{Anakotmai}[
  FontFace = {mb}{n}{ Font = *-Medium },
  FontFace = {l}{n}{ Font = *-Light },
]

\newfontfamily\TUFont{TUFont}
\newfontfamily\Pridi{Pridi}[
  FontFace = {el}{n}{ Font = *-ExtraLight },
  FontFace = {l}{n}{ Font = *-Light },
  FontFace = {mb}{n}{ Font = *-Medium },
  FontFace = {sb}{n}{ Font = *-Semibold },
]


\newfontfamily\thaifont{Norasi}
\newfontfamily\thaifontsf{Garuda}
\newfontfamily\thaifonttt{TlwgTypist}

\setdefaultlanguage[numerals=thai]{thai}
 \XeTeXlinebreaklocale "th"
\XeTeXlinebreakskip = 0pt plus 1pt \relax 


%% Command for test Thai Fonts

\def\textyamakkan{\symbol{"0E4E}}
\def\textfongmun{\symbol{"0E4F}}
\def\textangkhankhu{\symbol{"0E5A}}
\def\textkhomut{\symbol{"0E5B}}

\makeatletter
\def\flist{Laksaman,ChulabhornLikitText,Anakotmai,TUFont,Pridi}

\newcommand\showthaifont{
  \begin{itemize}
    \@for\next:=\flist%
  \do{%
  \item  \fontspec{\next} \fontseries{m}\fontshape{n}\selectfont แบบอักษรไทย \next{} ใน \LaTeX %
    }
  \end{itemize}
  }
\makeatother

\newcommand{\testthaipoem}[3]{%
#1\fontseries{#2}\fontshape{#3}\selectfont 
  \noindent
  \begin{tabbing}
  {{#1\fontseries{b}\fontshape{n}\selectfont
    XXXXXXXXXXXXXXXXXXXXXXXXXXXXX}} \=
  {{#1\fontseries{b}\fontshape{n}\selectfont
    XXXXXXXXXXXXXXXXXXXXXXXXXXXXX}}\kill
  \hspace{1em}๏ เป็นมนุษย์สุดประเสริฐเลิศคุณค่า \> กว่าบรรดาฝูงสัตว์เดรัจฉาน \\
  จงฝ่าฟันพัฒนาวิชาการ          \> อย่าล้างผลาญฤๅเข่นฆ่าบีฑาใคร\\
  ไม่ถือโทษโกรธแช่งซัดฮึดฮัดด่า    \> หัดอภัยเหมือนกีฬาอัชฌาสัย \\
  ปฏิบัติประพฤติกฎกำหนดใจ       \> พูดจาให้จ๊ะ ๆ จ๋า ๆ น่าฟังเอย ฯ\\
  \end{tabbing}}

\newcommand{\testenglish}[3]{%
#1 \fontseries{#2}\fontshape{#3}\selectfont
\noindent
A quick brown fox jumps over the lazy dog.}

\newcommand{\testEnglish}[3]{%
#1 \fontseries{#2}\fontshape{#3}\selectfont
  \noindent
  \MakeUppercase{A quick brown fox jumps over the lazy dog.}}

\newcommand{\testligkern}[3]{%
#1 \fontseries{#2}\fontshape{#3}\selectfont
\noindent

ที่ ท่า ทิ้ง ท้า กิ๊ง ก๊ง ตี๋ ต๋า บ่น ป่น, บ้น ป้น, บ๊น ป๊น, บ๋น ป๋น บิน ปิน บีน ปีน บิ่น ปิ่น บัน ปั่น บั่น
ก็ ป็ ปู่ ญ ญุ ญู ญฺ ฐ ฐุ ฐู ฐฺ กุ ฎุ ฎู ฎฺ ฏุ ฏู ฏฺ บำ บ่ำ ปำ ป่ำ -\textyamakkan{}
\textfongmun{} \textangkhankhu{} \textkhomut{}
ปะเฺติ็ลฺ โฺญฺ็จฺ ปั็วฮฺ ทฺ็อง เปฺิ็ว มูํย
แต็่ง เจฺํอ เปรฺิ่ห์ โจ๊่ เปฺี่ย โฺทร ม็่อง เติ็ง อาื ยาึ ปิํปี็ป็่ป๊่ปฺ่
จือรฺุ การฺู
- -- --- `` '' \dag{} \ddag{} \S{} \P{} \${} \ae{} \AE{} \oe{} \OE{} \aa{}
\AA{} \ss{} \copyright{} \textregistered{} \texttrademark{} \textbackslash{}
\textasciicircum{} \textasciitilde{} \textbar{} \textbraceleft{}
\textbraceright{} ?` !` ff fi fl ffi ffl tt ti AV\\

}

\newcommand{\testpali}[4]{%

\def\args{#2}
\def\pridiTxt{Pridi}
\def\chulabhornTxt{ChulabhornLikitText}
\def\anakotmaiTxt{Anakotmai}


\ifthenelse{\equal{#2}{Pridi}}%
      {\renewfontfamily#1{#2}[%
      FontFace = {el}{n}{ Font = *-ExtraLight },
      FontFace = {l}{n}{ Font = *-Light },
      FontFace = {mb}{n}{ Font = *-Medium },
      FontFace = {sb}{n}{ Font = *-Semibold },
      Script=Thai,
      Language=Pali
    ]}{
 \ifthenelse{\equal{#2}{Anakotmai}}%
   { \renewfontfamily#1{#2}[%
    FontFace = {l}{n}{ Font = *-Light },
    FontFace = {mb}{n}{ Font = *-Medium},
    Script=Thai,
    Language=Pali
  ]}{
    \ifthenelse{\equal{#2}{ChulabhornLikitText}}%
    {\renewfontfamily#1{#2}[%
    FontFace = {l}{n}{ Font = *-Light },
    FontFace = {mb}{n}{ Font = *-Medium},
    Script=Thai,
    Language=Pali
  ]}{
   \renewfontfamily#1{#2}[Script=Thai,Language=Pali]%
}
  }
    }

 
#1 \fontseries{#3}\fontshape{#4}\selectfont
\noindent
หตฺเถสุ ภิกฺขเว สติ, อาทานนิกฺเขปนํ ปญฺญายติ\\
เอวเมว โข ภิกฺขเว\\
จกฺขุสมิํปิ สติ\\
จกฺขุสมฺผสฺสปจฺจยา อุปฺปชฺชติ อชฺฌตฺตํ สุขทุกฺขํ\\
ทิฏฺฐา มยา ภิกฺขเว ฉ ผสฺสายตนิกา นาม นิรยา\\}

\begin{document}
\title{แบบอักษรภาษาไทย}
\subtitle{TrueType \& OpenType Font \& Fontspec \& \XeLaTeX }
\author{}
\date{\today}

\pagestyle{empty}
\maketitle

\pagestyle{empty}
\vfil


\begin{figure*}
  \Large
  \showthaifont
\end{figure*}
\vfil
\clearpage

\pagestyle{plain}

\section{\fontspec{Laksaman}\fontseries{b}\fontshape{n}\selectfont Laksaman -- ลักษมัณ\protect\footnote{ดัดแปลงจาก TH Sarabun New ของคุณศุภกิจ เฉลิมลาภ}}

\subsection{ตัวอย่างประโยคภาษาไทย}

\testthaipoem{\laksaman}{m}{n}

\testthaipoem{\laksaman}{b}{n}

\testthaipoem{\laksaman}{m}{it}

\testthaipoem{\laksaman}{b}{it}

\subsection{ตัวอย่างภาษาอังกฤษ}

\testenglish{\laksaman}{m}{n}

\testenglish{\laksaman}{b}{n}

\testenglish{\laksaman}{m}{it}

\testenglish{\laksaman}{b}{it}

\testEnglish{\laksaman}{m}{n}

\testEnglish{\laksaman}{b}{n}

\testEnglish{\laksaman}{m}{it}

\testEnglish{\laksaman}{b}{it}


\subsection{การจัดระดับตัวอักษรและตัวอักษรพิเศษ}

\testligkern{\laksaman}{m}{n}

\testligkern{\laksaman}{b}{n}

\testligkern{\laksaman}{m}{it}

\testligkern{\laksaman}{b}{it}


\subsection{ภาษาบาลี-สันสกฤต}

\testpali{\laksaman}{Laksaman}{m}{n}

\testpali{\laksaman}{Laksaman}{b}{n}

\testpali{\laksaman}{Laksaman}{m}{it}

\testpali{\laksaman}{Laksaman}{b}{it}

\vfil\pagebreak

\section{\ChulabhornLikitText\fontseries{b}\fontshape{n}\selectfont ChulabhornLikitText -- จุฬาภรณ์ลิขิต\protect\footnote{โดย Promphan Suksumek | Cadson Demak Team.}}

\subsection{ตัวอย่างประโยคภาษาไทย}

\testthaipoem{\ChulabhornLikitText}{l}{n}

\testthaipoem{\ChulabhornLikitText}{m}{n}

\testthaipoem{\ChulabhornLikitText}{mb}{n}

\testthaipoem{\ChulabhornLikitText}{b}{n}


\subsection{ตัวอย่างภาษาอังกฤษ}

\testenglish{\ChulabhornLikitText}{l}{n}

\testenglish{\ChulabhornLikitText}{m}{n}

\testenglish{\ChulabhornLikitText}{mb}{n}

\testenglish{\ChulabhornLikitText}{b}{n}

\testEnglish{\ChulabhornLikitText}{l}{n}

\testEnglish{\ChulabhornLikitText}{m}{n}

\testEnglish{\ChulabhornLikitText}{mb}{n}

\testEnglish{\ChulabhornLikitText}{b}{n}

\subsection{การจัดระดับตัวอักษรและตัวอักษรพิเศษ}

\testligkern{\ChulabhornLikitText}{l}{n}

\testligkern{\ChulabhornLikitText}{m}{n}

\testligkern{\ChulabhornLikitText}{mb}{n}

\testligkern{\ChulabhornLikitText}{b}{n}


\subsection{ภาษาบาลี-สันสกฤต}

\testpali{\ChulabhornLikitText}{ChulabhornLikitText}{l}{n}

\testpali{\ChulabhornLikitText}{ChulabhornLikitText}{m}{n}

\testpali{\ChulabhornLikitText}{ChulabhornLikitText}{mb}{n}

\testpali{\ChulabhornLikitText}{ChulabhornLikitText}{b}{n}


\vfil\pagebreak

\section{\Anakotmai\fontseries{b}\fontshape{n}\selectfont Anakotmai -- อนาคตใหม่\protect\footnote{โดย Smich Smanloh | Cadson Demak Co.,Ltd.}}

\subsection{ตัวอย่างประโยคภาษาไทย}

\testthaipoem{\Anakotmai}{l}{n}

\testthaipoem{\Anakotmai}{mb}{n}

\testthaipoem{\Anakotmai}{b}{n}


\subsection{ตัวอย่างภาษาอังกฤษ}

\testenglish{\Anakotmai}{l}{n}

\testenglish{\Anakotmai}{mb}{n}

\testenglish{\Anakotmai}{b}{n}

\testEnglish{\Anakotmai}{l}{n}

\testEnglish{\Anakotmai}{mb}{n}

\testEnglish{\Anakotmai}{b}{n}

\subsection{การจัดระดับตัวอักษรและตัวอักษรพิเศษ}

\testligkern{\Anakotmai}{l}{n}

\testligkern{\Anakotmai}{mb}{n}

\testligkern{\Anakotmai}{b}{n}


\subsection{ภาษาบาลี-สันสกฤต}

\testpali{\Anakotmai}{Anakotmai}{l}{n}

\testpali{\Anakotmai}{Anakotmai}{mb}{n}

\testpali{\Anakotmai}{Anakotmai}{b}{n}


\vfil\pagebreak

\section{\TUFont\fontseries{b}\fontshape{n}\selectfont TUFont -- ธรรมศาสตร์\protect\footnote{โดย Umaporn Matturos}}

\subsection{ตัวอย่างประโยคภาษาไทย}

\testthaipoem{\TUFont}{m}{n}

\testthaipoem{\TUFont}{b}{n}


\subsection{ตัวอย่างภาษาอังกฤษ}

\testenglish{\TUFont}{m}{n}

\testenglish{\TUFont}{b}{n}

\testEnglish{\TUFont}{m}{n}

\testEnglish{\TUFont}{b}{n}


\subsection{การจัดระดับตัวอักษรและตัวอักษรพิเศษ}

\testligkern{\TUFont}{m}{n}

\testligkern{\TUFont}{b}{n}


\subsection{ภาษาบาลี-สันสกฤต}

\testpali{\TUFont}{TUFont}{m}{n}

\testpali{\TUFont}{TUFont}{b}{n}


\vfil\pagebreak

\section{\Pridi\fontseries{b}\fontshape{n}\selectfont Pridi -- ปรีดี\protect\footnote{โดย Katatrad Team | Cadson Demak}}

\subsection{ตัวอย่างประโยคภาษาไทย}

\testthaipoem{\Pridi}{el}{n}

\testthaipoem{\Pridi}{l}{n}

\testthaipoem{\Pridi}{m}{n}

\testthaipoem{\Pridi}{mb}{n}

\testthaipoem{\Pridi}{sb}{n}

\testthaipoem{\Pridi}{b}{n}


\subsection{ตัวอย่างภาษาอังกฤษ}

\testenglish{\Pridi}{el}{n}

\testenglish{\Pridi}{l}{n}

\testenglish{\Pridi}{m}{n}

\testenglish{\Pridi}{mb}{n}

\testenglish{\Pridi}{sb}{n}

\testenglish{\Pridi}{b}{n}

\testEnglish{\Pridi}{el}{n}

\testEnglish{\Pridi}{l}{n}

\testEnglish{\Pridi}{m}{n}

\testEnglish{\Pridi}{mb}{n}

\testEnglish{\Pridi}{sb}{n}

\testEnglish{\Pridi}{b}{n}

\subsection{การจัดระดับตัวอักษรและตัวอักษรพิเศษ}

\testligkern{\Pridi}{el}{n}

\testligkern{\Pridi}{l}{n}

\testligkern{\Pridi}{m}{n}

\testligkern{\Pridi}{mb}{n}

\testligkern{\Pridi}{sb}{n}

\testligkern{\Pridi}{b}{n}




\subsection{ภาษาบาลี-สันสกฤต}

\testpali{\Pridi}{Pridi}{el}{n}

\testpali{\Pridi}{Pridi}{l}{n}

\testpali{\Pridi}{Pridi}{m}{n}

\testpali{\Pridi}{Pridi}{mb}{n}

\testpali{\Pridi}{Pridi}{sb}{n}

\testpali{\Pridi}{Pridi}{b}{n}


\vfil\pagebreak
\end{document}